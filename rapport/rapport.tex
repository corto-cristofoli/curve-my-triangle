\documentclass{article}

\usepackage[T1]{fontenc}
\usepackage[french]{babel}
\usepackage[autolanguage]{numprint} % for the \nombre command

\usepackage{hyphenat}
\hyphenation{mate-mática recu-perar}

\usepackage{graphicx}
\graphicspath{ {./images} }

\title{Rapport CGDI}
\author{Corto Cristofoli et Max Royer}


\begin{document}
\maketitle
\tableofcontents



\section{Introduction}
Cet article est réalisé dans le cadre de notre cours de CGDI.
Il s'agit de la présentation de notre implémentation de l'article
de recherche \textit{Curved PN triangle}, par Alex Vlachos, Jorg Peters,
Chas Boyd et Jason L. Mitchell.

L'idée de l'article est de présenter un algorithme de lissage de modèles 3D de
basse résolution. Pour cela, l'idée est d'utiliser la méthode des « Curved PN
triangles ».


\section{Principe de l'algorithme}

\section{Indexation des sommets du modèle 3D}
% Où tu peux expliquer comment fonctionne `geometry-central` avec uniquement la
% possibilité d'ajouter des sommets à des faces (ou edges) ce qui fait
% trianguler automatiquement. D'où l'importance de repartir de zero.

% TODO: pas oublier de produire des visuels !

\end{document}
