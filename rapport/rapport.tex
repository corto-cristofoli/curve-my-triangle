\documentclass{article}

\usepackage[T1]{fontenc}
\usepackage[french]{babel}
\usepackage[autolanguage]{numprint} % for the \nombre command

\usepackage{hyphenat}
\hyphenation{mate-mática recu-perar}

\usepackage{graphicx}
\graphicspath{ {./images} }

\def\Cpp{C\texttt{++} }

\title{Rapport CGDI}
\author{Corto Cristofoli et Max Royer}


\begin{document}
\maketitle
% \tableofcontents



\section{Introduction}
Cet article est réalisé dans le cadre de notre cours de CGDI.
Il s'agit de la présentation de notre implémentation de l'article
de recherche \textit{Curved PN triangle}, par Alex Vlachos, Jorg Peters,
Chas Boyd et Jason L. Mitchell.

L'article présente un algorithme de lissage de modèles 3D de basse résolution.
Pour cela, l'idée est d'utiliser la méthode des « Curved PN triangles ». Un
modèle 3D de basse résolution n'a que peu de polygones et donc une alure
nécessairement anguleuse. Pour le rendre plus organique un lissage est
nécessaire : il faut rajouter un grand nombre de triangle afin de faire
disparaitre ces angles. Ce rajout se fait algorithmiquement en suivant des
surfaces de bézier triangulaires.

Si l'article décrit l'algorithme de manière théorique, nous nous sommes penchés
sur sa compréhension et son implémentation en \Cpp à l'aide de la
bibliothèque \textbf{geometry-central}.


\section{Principe de l'algorithme}
Dans tout le rapport nous considérons qu'un modèle 3D est un ensemble de face
triangulaires.

\section{Enjeux d'implémentation de \textit{Geometry Central}}
% Où on explique comment fonctionne `geometry-central` avec uniquement la
% possibilité d'ajouter des sommets à des faces (ou edges) ce qui fait
% trianguler automatiquement. D'où l'importance de repartir de zero.
% (je peux le faire ça)

\section{Indexation des sommets du modèle 3D}
% Où tu peux parler des unordered set et tout

% TODO: pas oublier de produire des visuels !

\end{document}
